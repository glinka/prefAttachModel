\documentclass[12pt]{article}

\usepackage{graphicx}
%\usepackage{subcaption}
\usepackage{amsfonts}
\usepackage{amsmath}
\usepackage{mathtools}
\usepackage{epstopdf}
%\usepackage{fullpage}

\begin{document}

\begin{flushright}
From: A. Holiday, I. Kevrekidis
\end{flushright}

\vspace{5mm}

% \begin{flushleft}
% To: Terre Arena \\
% Senior Editorial Assistant \\
% Physical Review E \\
% Email: pre@aps.org \\
% http://journals.aps.org/pre/
% \end{flushleft}

\vspace{2mm}

\section*{Response to comments/changes implemented in Equation-free analysis
of a dynamically evolving multigraph by A. Holiday and I. Kevrekidis.}

First, we would like to thank both reviewers for their helpful input!

\vspace{3mm}

\textbf{First Editor's recommendations}
\vspace*{\baselineskip}

* {\em ... I would like to make a few suggestions to the authors of
  the paper for a possible extension of their methodology in the
  investigation of more realistic models of complex networks and
  possibly models supported by concrete empirical data. The notion of
  a multigraph is too general to include the whole range of
  applications of graph theory in complex network science. As a matter
  of fact, a further step would be towards the consideration of
  attributes (or labels) on nodes and links of a (multi)graph... }.

\textbf{OUR REPLY}: It is a very good suggestion to apply the method
to other systems with labeled nodes, and we have in fact already
addressed a first example of this in \cite{kattis_modeling_2015} in which
an SIS epidemiological model is investigated with similar
techniques. We have added a reference to this paper in the text.
\vspace*{\baselineskip}

* {\em ... in the literature of network science there are many
  examples of such large empirical networks exhibiting structures of a
  multilayer (or multiplex) typology. Equation-free modeling applied
  to such empirically sustained networks would be expected to
  demonstrate a very interesting novel attainment in the evolving
  field of complex network science.}

\textbf{OUR REPLY}: An application to empirically-based network models
would also be worthwhile, but as the aim of the current paper is
simply to show the extension of our techniques to systems with
multiple edges, we believe the recommendation would serve well as a
separate, second paper on the topic which we are already working on.
\vspace*{\baselineskip}

\textbf{Second Editor's recommendations}

* {\em While it is clear that the paper deals with new techniques
  applied in well known systems, it is not clear what questions it
  tries to answer, what are the objectives of this work. Thus, I would
  try to add a couple of sentences both in the abstract and in the
  conclusion sections showing where such results are useful, other
  that as theoretical techniques.}

\textbf{OUR REPLY}: To clarify the prospects for future application
areas, we have added a little information in the abstract and
conclusion section mentioning relevant systems that could benefit from
such a study.  \vspace*{\baselineskip}

{\em There is a Figure 1, but it is not discussed in the text of the
  paper.  Figure 2. It is not clear which is the independent
  variable. Is it the degree or is it the vertex number? Are the
  numbers given here summations of many simulations?  How many?}

\textbf{OUR REPLY}: We have added a missing reference to Fig. 1, and
appreciate the reviewer's help in noticing its absence. To clarify
Fig. 2 we have created a new caption. Indeed the results are an
instantaneous average over twenty simulations. Also, the vertex number
is simply a pseudo-labeling of each node for the purpose of
plotting. In the actual model, vertices are not labeled.
\vspace*{\baselineskip}

{\em It is clear here that multiple links are allowed between two
  specific nodes.  While this may be a mathematical notion, all
  practical problems in networks today deal with a single link between
  two nodes. Would the results be the same for such a case?}

\textbf{OUR REPLY}: The model could also be cast as a weighted
network, in which the edge weight signifies the number of connections
between two nodes. We now note this in the model description. While
this might be a more familiar formulation to some researchers, since
previous theoretical results on the system were couched in the
language of multiple edges, we chose this framework. 
\vspace*{\baselineskip}

\bibliographystyle{abbrv} \bibliography{multigraph-refs.bib}

\end{document}