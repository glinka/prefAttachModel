\documentclass[11pt]{article}
\usepackage{graphicx, subcaption}
\usepackage[top=1in, bottom=1in, left=1in, right=1in]{geometry}
\graphicspath{ {./figs/} }
\pagestyle{plain}
\def\changemargin#1#2{\list{}{\rightmargin#2\leftmargin#1}\item[]}
\let\endchangemargin=\endlist 
\setlength\parindent{0cm}
\begin{document}
\title{\vspace{-5mm}Differences between canonical preferential attachment model and Bal\'{a}zs'}
\maketitle

The canonical preferential attachment model, described in the attached paper and called the ``Bar\'{a}basi-Albert model'', describes a method of \textbf{constructing} a network that will have a power-law degree distribution. The method is simple:

\begin{enumerate}
\item Begin with a complete graph on $m_0$ vertices
\item Add an additional $m$ vertices which will be connected to $k \leq m_0$ of the existing vertices with probability $P_i = \frac{d_i}{\sum\limits_j d_j}$ where $d_i$ is the degree of $v_i$
\end{enumerate}

Bal\'{a}zs' model, fully called the ``edge-conservative preferential attachment model'' describes, of course, a method of \textbf{evolving} a network. At each step, the probability that an edge is added to vertex $i$ is given by $P_i = \frac{d_i + \kappa}{\sum\limits_i d_i + n \kappa}$. \\ \\ \\

{\huge Thus when they hear ``preferential attachment'', people may think of a network \textbf{construction} algorithm, and may be familiar with $P_i = \frac{d_i}{\sum\limits_j d_j}$ instead of $P_i = \frac{d_i + \kappa}{\sum\limits_j d_j + n \kappa}$.}

%% \begin{changemargin}{0.0cm}{0.0cm}
%% \centering

%% \centering
\begin{figure}[h!]
  \vspace{-30mm}
  \begin{subfigure}{0.5\textwidth}
    \includegraphics[height=60mm]{n_500_n3_3d_v2}
    \caption{$n^{3}$ timescale}
    \label{fig:100s3}
  \end{subfigure}%
  \begin{subfigure}{0.5\textwidth}
    \includegraphics[height=60mm]{n_500_2n3_3d_v2}
    \caption{$n^{2}$ timescale}
    \label{fig:100s3}
  \end{subfigure}%
  \caption{Evolution of degrees. ($n=500$, $m=250000$)}
  \label{fig:n500_3d}
\end{figure}

\begin{figure}[h!]
  \vspace{-5mm}
  \centering
  \begin{subfigure}{0.5\textwidth}
    \centering
    \includegraphics[height=50mm]{n_500_n3_deg_percentile}
    \caption{$n^{3}$ timescale}
    \label{fig:100s3}
  \end{subfigure}%
  \begin{subfigure}{0.5\textwidth}
    \centering
    \includegraphics[height=50mm]{n_500_2n3_deg_percentile}
    \caption{$n^{2}$ timescale}
    \label{fig:100s3}
  \end{subfigure}%
  \caption{Evolution of degrees and percentiles, a projection of Fig. \ref{fig:n500_3d} along the time axis. ($n=500$, $m=250000$)}
\end{figure}

\begin{figure}[h!]
  \vspace{-5mm}
  \centering
  \begin{subfigure}{0.5\textwidth}
    \centering
    \includegraphics[height=60mm]{n_500_n3_degchange_percentile}
    \caption{$n^{3}$ timescale}
    \label{fig:100s3}
  \end{subfigure}%
  \begin{subfigure}{0.5\textwidth}
    \centering
    \includegraphics[height=60mm]{n_500_2n3_degchange_percentile}
    \caption{$n^{2}$ timescale}
    \label{fig:100s3}
  \end{subfigure}%
  \caption{Change in degree distribution between each step plotted against percentiles. ($n=500$, $m=250000$)}
\end{figure}

\begin{figure}[h!]
  \vspace{-5mm}
  \begin{subfigure}{0.5\textwidth}
    \includegraphics[height=50mm]{n_500_n3_deg_step}
    \caption{$n^{3}$ timescale}
    \label{fig:100s3}
  \end{subfigure}%
  \begin{subfigure}{0.5\textwidth}
    \includegraphics[height=50mm]{n_500_2n3_deg_step}
    \caption{$n^{2}$ timescale}
    \label{fig:100s3}
  \end{subfigure}%
  \caption{Evolution of degree distribution. Color indicates percentile, i.e. the median degree at each step is colored green. ($n=500$, $m=250000$)}
\end{figure}

%% \begin{figure}[h!]
%%   \centering
%%   \includegraphics[height=60mm]{n_500_short}
%%   \caption{$n^{2}$ timescale, $n=500$, $m=250000$}
%%   \label{fig:500sv}
%% \end{figure}
%% \begin{figure}[h!]
%%   \centering
%%   \includegraphics[height=60mm]{n_500_long_3d}
%%   \caption{$n^{3}$ timescale, $n=500$, $m=250000$}
%%   \label{fig:500l3}
%% \end{figure}
%% \begin{figure}[h!]
%%   \centering
%%   \includegraphics[height=60mm]{n_500_long_time}
%%   \caption{$n^{3}$ timescale, $n=500$, $m=250000$}
%%   \label{fig:500lt}
%% \end{figure}
%% \begin{figure}[h!]
%%   \centering
%%   \includegraphics[height=60mm]{n_500_long}
%%   \caption{$n^{3}$ timescale, $n=500$, $m=250000$}
%%   \label{fig:500lv}
%% \end{figure}

%% \begin{figure}[h!]
%%   \centering
%%   \includegraphics[height=60mm]{n_1000_short_3d}
%%   \caption{$n^{2}$ timescale, $n=1000$, $m=1000000$}
%%   \label{fig:1000s3}
%% \end{figure}
%% \begin{figure}[h!]
%%   \centering
%%   \includegraphics[height=60mm]{n_1000_short_time}
%%   \caption{$n^{2}$ timescale, $n=1000$, $m=1000000$}
%%   \label{fig:1000st}
%% \end{figure}
%% \begin{figure}[h!]
%%   \centering
%%   \includegraphics[height=60mm]{n_1000_short}
%%   \caption{$n^{2}$ timescale, $n=1000$, $m=1000000$}
%%   \label{fig:1000sv}
%% \end{figure}
%% \begin{figure}[h!]
%%   \centering
%%   \includegraphics[height=60mm]{n_1000_long_3d}
%%   \caption{$n^{3}$ timescale, $n=1000$, $m=1000000$}
%%   \label{fig:1000l3}
%% \end{figure}
%% \begin{figure}[h!]
%%   \centering
%%   \includegraphics[height=60mm]{n_1000_long_time}
%%   \caption{$n^{3}$ timescale, $n=1000$, $m=1000000$}
%%   \label{fig:1000lt}
%% \end{figure}
%% \begin{figure}[h!]
%%   \centering
%%   \includegraphics[height=60mm]{n_1000_long}
%%   \caption{$n^{3}$ timescale, $n=1000$, $m=1000000$}
%%   \label{fig:1000lv}
%% \end{figure}


%% \end{changemargin}

\end{document}
