\documentclass[11pt]{article}
\setlength{\textheight}{240mm}
%\setlength{\voffset}{0mm}
\addtolength{\topmargin}{-30mm}
\setlength{\textwidth}{155mm}
\setlength{\oddsidemargin}{5mm}
\usepackage{graphicx, subfig}
\pagestyle{plain}
\begin{document}
\title{Preferential Attachment Coarse Projective Integration}
\author{Alexander Holiday\vspace{-2ex}}
\date{09/01/2013}
\maketitle
\section*{Introduction}
The preferential attachment model, detailed in \cite{balasz:rsa12}, was implemented in C++. The system naturally converges to a limiting ``graphon'' (see \cite{lovasz:jcombth06}), and in order to increase the rate of convergence, coarse projective integration was implemented. This brief outlines the model dynamics and reviews current simulation progress.
\section*{Preferential Attachment Model}
The initial state of the preferential attachment model is an Erd\H{o}s-R\'{e}nyi random graph with $n$ vertices and $m\sim O(n^{2})$ edges. The system evolves as a discrete-time Markov chain, in which the following actions are performed during each step:
\begin{enumerate}
\item An edge $e_{old}$ is chosen uniformly at random from the set of all possible edges, $E(G)$.
\item A vertex end of $e_{old}$, $v_{1}$, is chosen uniformly from the two ends.
\item $e_{old}$ is removed from the graph, and a vertex $v_{2}$ is chosen from $V(G)$ with a probability determined based on linear preferential attachment:
\[
P(v_{2}=v_{i})=\frac{deg(v_{i})+\kappa}{2m+n\kappa}
\]
where \kappa is a model parameter and $deg(v_{i})$ denotes the degree of vertex $i$ in $V$.
\item An edge $e_{new}$ is added between $v_{1}$ is connected to $v_{2}$.
\end{enumerate}
Two distinct timescales arise from these dynamics, $T\asymp n^{2}$ and $T\asymp n^{3}$ where $T$ is the number of steps.
\end{document}

